\documentclass{article}

% Language setting
% Replace `english' with e.g. `spanish' to change the document language
\usepackage[english]{babel}

% Set page size and margins
% Replace `letterpaper' with `a4paper' for UK/EU standard size
\usepackage[letterpaper,top=2cm,bottom=2cm,left=3cm,right=3cm,marginparwidth=1.75cm]{geometry}

% Useful packages
\usepackage{amsmath}
\usepackage{graphicx}
\usepackage[table]{xcolor}
\usepackage[colorlinks=true, allcolors=blue]{hyperref}

\usepackage[
backend=biber,
style=ieee,
]{biblatex}


\addbibresource{sample.bib} %Imports bibliography file

\begin{document}

% ---------------------------
% Title page (custom)
% ---------------------------
\begin{titlepage}
  \begin{center}
    \includegraphics[width=5cm]{nup_logo.png} \\[1cm]
    {\Large Neapolis University Pafos} \\[0.5cm]

    \begin{tabular}{@{}l@{}}
      {\large \textbf{Course Code:} IS507} \\[0.2cm]
      {\large \textbf{Course Title:} Disruptive Technologies} \\[0.2cm]
      {\large \textbf{Audience Instructor:} Georgios Sklias}
    \end{tabular} \\[2cm]

    {\huge \textbf{EU Funding Concept Note for Your Start-Up}} \\[1.5cm]

     \begin{tabular}{@{}l@{}}
      {\large \textbf{Name:} Aleksandr Petrunin} \\[0.3cm]
      {\large \textbf{Student ID:} 1251114137}
    \end{tabular} \\[2cm]

    {\large \today}
  \end{center}
\end{titlepage}



\section{Introduction}
Search for an EU programme/call that:
Is currently open (or was open recently), and
Is realistically relevant to your start-up’s sector and activities.
Examples of funding programmes (non-exhaustive): Horizon Europe, Erasmus+, Digital Europe, LIFE, Interreg, etc.

\subsection{Startup}
\paragraph{Sector:}
IT Solutions for Smart Data Analytics in Sustainable Country Development
\paragraph{Target Market:}
EU-countries governments and related public sector organisations
\paragraph{Products/Services:}
AI-powered Data Analytics Platform for Environmental and Urban Planning
\paragraph{Activities:}
Development, Deployment, and Support of AI-driven Data Analytics Solutions
\paragraph{Unique Selling Proposition (USP):}
Cutting-edge AI technology tailored for sustainable development needs of public sector entities
\subsection{Selected Call}
\href{https://interreg.eu/calls-for-projects/interreg-romania-bulgaria-opens-a-call-for-projects-of-strategic-importance/
}{https://interreg.eu/calls-for-projects/interreg-romania-bulgaria-opens-a-call-for-projects-of-strategic-importance/}
\\
\paragraph{Programme:} Interreg Romania-Bulgaria.
Call 6 - Call dedicated to the operations of strategic importance addressing the navigability and rail infrastructure.
\subsection{Relevance to Startup}
The selected call aligns with our startup's focus on providing IT solutions for sustainable country development. By leveraging our AI-powered data analytics platform, we can contribute to enhancing the navigability and rail infrastructure between Romania and Bulgaria. Our technology can help analyze environmental impact, optimize routes, and improve overall infrastructure planning, making us a suitable candidate for this funding opportunity.

\newpage
\section{Concept note: Interreg Romania-Bulgaria Call 6}
\paragraph{Title:} AI-Driven Data Analytics for Enhancing Navigability and Rail Infrastructure between Romania and Bulgaria

\paragraph{Funding Programme Deadline:} 22nd December 2025
 
\paragraph{Priorities Addressed by the project} A well-connected region.
 
 
\paragraph{Description of the Idea} Include background information and an assessment of the particular needs or challenges of the target group.

\textbf{Background:} The Romania-Bulgaria cross-border region faces significant challenges in developing sustainable, climate-resilient, and intelligent transport infrastructure, particularly concerning navigability along the Danube River and rail connectivity. Current infrastructure planning lacks integrated data analytics capabilities, resulting in suboptimal resource allocation, environmental impact assessments, and strategic decision-making.

\textbf{Assessment of Needs:} The target group—comprising Romanian and Bulgarian governmental bodies, transport authorities, and infrastructure agencies—requires advanced technological solutions to:
\begin{itemize}
    \item Analyze environmental impact of infrastructure projects on ecosystems and climate
    \item Optimize route planning for both rail and waterway transport
    \item Predict infrastructure maintenance needs using predictive analytics
    \item Integrate multimodal transport data for enhanced TEN-T network connectivity
    \item Support evidence-based policy decisions with real-time data visualization
    \item Ensure compliance with EU climate resilience and sustainability standards
\end{itemize}

\textbf{Challenges:} Key challenges include fragmented data sources, lack of cross-border data integration, limited AI/ML expertise in public sector organizations, and insufficient tools for scenario modeling and impact assessment. Our AI-powered platform addresses these gaps by providing an integrated, intelligent solution tailored to the specific requirements of cross-border infrastructure development.
 
 
\paragraph{Objectives} Make sure they (a) Respond to the selected priorities and (b) Respond to the needs or challenges specified above

\textbf{Objective 1:} Develop and deploy an AI-powered data analytics platform specifically designed for cross-border infrastructure planning, focusing on navigability and rail connectivity between Romania and Bulgaria, contributing to Specific Objective 3.2 (sustainable, climate-resilient, intelligent mobility).

\textbf{Objective 2:} Enhance decision-making capacity of Romanian and Bulgarian transport authorities through advanced data visualization, predictive modeling, and scenario analysis tools that integrate environmental, economic, and social impact assessments.

\textbf{Objective 3:} Improve access to TEN-T networks by optimizing intermodal connectivity between rail and waterway infrastructure through AI-driven route optimization and capacity analysis.

\textbf{Objective 4:} Build institutional capacity in AI and data analytics within cross-border public sector organizations through knowledge transfer, training programs, and establishment of sustainable data-sharing protocols.

\textbf{Objective 5:} Demonstrate measurable improvements in infrastructure planning efficiency, environmental sustainability, and cross-border cooperation through pilot implementation and comprehensive monitoring of key performance indicators
 

\paragraph{Activities}
Describe the activities that the project will undertake to produce the envisaged results

\textbf{Activity 1: Needs Assessment and Data Collection (Months 1-6)}
\begin{itemize}
    \item Conduct comprehensive stakeholder consultations in Romania and Bulgaria
    \item Map existing data sources and infrastructure planning processes
    \item Define technical requirements and performance indicators
    \item Establish data-sharing agreements and protocols
\end{itemize}

\textbf{Activity 2: Platform Development and Customization (Months 4-18)}
\begin{itemize}
    \item Develop core AI/ML algorithms for infrastructure analysis
    \item Create integrated database architecture for cross-border data
    \item Build user-friendly dashboards and visualization tools
    \item Implement predictive maintenance and optimization modules
    \item Conduct iterative testing and refinement with end-users
\end{itemize}

\textbf{Activity 3: Pilot Implementation (Months 12-24)}
\begin{itemize}
    \item Deploy platform in selected pilot regions/corridors
    \item Apply analytics to specific navigability and rail projects
    \item Monitor performance and gather user feedback
    \item Conduct environmental impact assessments using the platform
\end{itemize}

\textbf{Activity 4: Capacity Building and Training (Months 6-30)}
\begin{itemize}
    \item Design and deliver training programs for public sector staff
    \item Organize cross-border knowledge exchange workshops
    \item Develop user manuals and technical documentation
    \item Establish helpdesk and ongoing support mechanisms
\end{itemize}

\textbf{Activity 5: Evaluation and Sustainability Planning (Months 24-36)}
\begin{itemize}
    \item Conduct comprehensive impact evaluation
    \item Document best practices and lessons learned
    \item Develop sustainability and scaling strategy
    \item Disseminate results to wider stakeholder community
\end{itemize}

\paragraph{Project results}
tangible deliverables of the project (such as
curricula, pedagogical and youth work materials, open educational resources, IT tools,  studies, peer-learning methods, etc.).

\textbf{R1. AI-Powered Infrastructure Analytics Platform:} Fully functional cloud-based platform with modules for data integration, predictive analytics, route optimization, environmental impact assessment, and real-time visualization dashboards. Platform includes API for integration with existing systems.

\textbf{R2. Cross-Border Infrastructure Database:} Comprehensive integrated database containing navigability data (Danube River), rail network information, environmental parameters, traffic flows, and socio-economic indicators for the Romania-Bulgaria cross-border region.

\textbf{R3. Feasibility Studies and Impact Assessments:} Three detailed studies: (1) Analysis of current infrastructure gaps and optimization opportunities; (2) Environmental and climate resilience impact assessment of proposed interventions; (3) Cost-benefit analysis of AI-driven planning versus traditional methods.

\textbf{R4. Training Materials and Capacity Building Package:} Complete training curriculum including user manuals, video tutorials, hands-on workshops, e-learning modules, and technical documentation in Romanian, Bulgarian, and English. Certification program for platform administrators.

\textbf{R5. Policy Recommendations and Sustainability Framework:} Strategic document outlining policy recommendations for AI adoption in infrastructure planning, data governance framework for cross-border cooperation, and sustainability plan for long-term platform operation and maintenance.
 

\paragraph{Key outcomes}
Make sure that the outcomes derive from the proposed Activities /Project Results and that, at the same time, answer the set objectives.

\textbf{Outcome 1: Enhanced Infrastructure Planning Efficiency}
\begin{itemize}
    \item Reduction of infrastructure planning time by 30-40\% through automated data analysis
    \item Improved accuracy of traffic flow predictions and capacity assessments by 25\%
    \item Enhanced cross-border coordination with 50\% faster data exchange between Romanian and Bulgarian authorities
\end{itemize}

\textbf{Outcome 2: Improved Environmental Sustainability}
\begin{itemize}
    \item 20\% reduction in environmental assessment time through AI-powered impact modeling
    \item Identification of climate-resilient infrastructure solutions with lower carbon footprint
    \item Better integration of environmental protection measures in infrastructure projects
\end{itemize}

\textbf{Outcome 3: Strengthened Institutional Capacity}
\begin{itemize}
    \item At least 120 public sector professionals trained in AI and data analytics
    \item Establishment of 4 cross-border working groups for ongoing collaboration
    \item Sustainable data-sharing protocols adopted by partner institutions
\end{itemize}

\textbf{Outcome 4: Improved TEN-T Network Connectivity}
\begin{itemize}
    \item Optimized multimodal transport routes connecting rail and waterway infrastructure
    \item Enhanced accessibility to TEN-T core network corridors
    \item Better integration of regional and local mobility with TEN-T networks
\end{itemize}

\textbf{Outcome 5: Evidence-Based Policy Development}
\begin{itemize}
    \item Data-driven strategic documents for infrastructure development adopted by authorities
    \item Improved resource allocation based on predictive analytics and scenario modeling
    \item Enhanced transparency and stakeholder engagement in planning processes
\end{itemize}
 
 \paragraph{Partners' profile} Describe any specific requirement e.g. type of organisation needed (NGO, VET, etc.), and/or country of establishment, and/or complementary skills that we are looking for in the partnership (e.g. expertise in IT tools or in working with children etc.)

\textbf{Lead Partner Requirements:}
\begin{itemize}
    \item Our startup (IT Solutions company) based in Romania or Bulgaria
    \item Expertise in AI, machine learning, and data analytics
    \item Experience in developing solutions for public sector or infrastructure projects
    \item Technical capacity for cloud-based platform development and maintenance
\end{itemize}

\textbf{Required Project Partners:}

\textbf{Partner 1: Romanian Transport Authority or Infrastructure Agency}
\begin{itemize}
    \item Public sector organization responsible for transport infrastructure planning
    \item Access to relevant data sources and decision-making processes
    \item Capacity to pilot and adopt the platform
    \item Commitment to long-term sustainability
\end{itemize}

\textbf{Partner 2: Bulgarian Transport Authority or Infrastructure Agency}
\begin{itemize}
    \item Counterpart to Romanian partner for cross-border coordination
    \item Similar mandate and capacity requirements
    \item Experience in EU-funded projects preferred
\end{itemize}

\textbf{Partner 3: Research Institution or University}
\begin{itemize}
    \item Expertise in transport engineering, environmental science, or data science
    \item Capacity for impact assessment and evaluation research
    \item Located in Romania or Bulgaria
    \item Experience in knowledge transfer and capacity building
\end{itemize}

\textbf{Partner 4: Environmental NGO or Sustainability Consultancy (Optional)}
\begin{itemize}
    \item Expertise in environmental impact assessment and climate resilience
    \item Knowledge of EU environmental standards and regulations
    \item Experience in stakeholder engagement and policy advocacy
\end{itemize}

\textbf{Complementary Skills Needed:}
\begin{itemize}
    \item GIS (Geographic Information Systems) expertise
    \item Knowledge of Danube navigation and inland waterway management
    \item Rail infrastructure technical expertise
    \item Project management and EU funding experience
    \item Multilingual capabilities (Romanian, Bulgarian, English)
    \item Stakeholder engagement and communication skills
\end{itemize}
 
 




\newpage
\printbibliography
\nocite{*}

\end{document}
