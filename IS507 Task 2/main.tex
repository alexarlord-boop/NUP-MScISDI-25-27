\documentclass{article}

% Language setting
% Replace `english' with e.g. `spanish' to change the document language
\usepackage[english]{babel}

% Set page size and margins
% Replace `letterpaper' with `a4paper' for UK/EU standard size
\usepackage[letterpaper,top=2cm,bottom=2cm,left=3cm,right=3cm,marginparwidth=1.75cm]{geometry}

% Useful packages
\usepackage{amsmath}
\usepackage{graphicx}
\usepackage[table]{xcolor}
\usepackage[colorlinks=true, allcolors=blue]{hyperref}

\usepackage[
backend=biber,
style=ieee,
]{biblatex}


\addbibresource{sample.bib} %Imports bibliography file

\begin{document}

% ---------------------------
% Title page (custom)
% ---------------------------
\begin{titlepage}
  \begin{center}
    \includegraphics[width=5cm]{nup_logo.png} \\[1cm]
    {\Large Neapolis University Pafos} \\[0.5cm]

    \begin{tabular}{@{}l@{}}
      {\large \textbf{Course Code:} IS507} \\[0.2cm]
      {\large \textbf{Course Title:} Disruptive Technologies} \\[0.2cm]
      {\large \textbf{Audience Instructor:} Georgios Sklias}
    \end{tabular} \\[2cm]

    {\huge \textbf{EU Funding Concept Note for Your Start-Up}} \\[1.5cm]

     \begin{tabular}{@{}l@{}}
      {\large \textbf{Name:} Aleksandr Petrunin} \\[0.3cm]
      {\large \textbf{Student ID:} 1251114137}
    \end{tabular} \\[2cm]

    {\large \today}
  \end{center}
\end{titlepage}



\section{Introduction}
Search for an EU programme/call that:
Is currently open (or was open recently), and
Is realistically relevant to your start-up’s sector and activities.
Examples of funding programmes (non-exhaustive): Horizon Europe, Erasmus+, Digital Europe, LIFE, Interreg, etc.

\subsection{Startup}
\paragraph{Sector:}
IT Solutions for Smart Data Analytics in Sustainable Country Development
\paragraph{Target Market:}
EU-countries governments and related public sector organisations
\paragraph{Products/Services:}
AI-powered Data Analytics Platform for Environmental and Urban Planning
\paragraph{Activities:}
Development, Deployment, and Support of AI-driven Data Analytics Solutions
\paragraph{Unique Selling Proposition (USP):}
Cutting-edge AI technology tailored for sustainable development needs of public sector entities
\subsection{Selected Call}
\href{https://interreg.eu/calls-for-projects/interreg-romania-bulgaria-opens-a-call-for-projects-of-strategic-importance/
}{https://interreg.eu/calls-for-projects/interreg-romania-bulgaria-opens-a-call-for-projects-of-strategic-importance/}
\\
\paragraph{Programme:} Interreg Romania-Bulgaria.
Call 6 - Call dedicated to the operations of strategic importance addressing the navigability and rail infrastructure.
\subsection{Relevance to Startup}
The selected call aligns with our startup's focus on providing IT solutions for sustainable country development. By leveraging our AI-powered data analytics platform, we can contribute to enhancing the navigability and rail infrastructure between Romania and Bulgaria. Our technology can help analyze environmental impact, optimize routes, and improve overall infrastructure planning, making us a suitable candidate for this funding opportunity.

\newpage
\section{Concept note: Interreg Romania-Bulgaria Call 6}
\paragraph{Title:} AI-Driven Data Analytics for Enhancing Navigability and Rail Infrastructure between Romania and Bulgaria

\paragraph{Funding Programme Deadline:} 22nd December 2025
 
\paragraph{Priorities Addressed by the project} A well-connected region.
 
 
\paragraph{Description of the Idea} Include background information and an assessment of the particular needs or challenges of the target group.

\textbf{Background:} The Romania-Bulgaria cross-border region faces significant challenges in developing sustainable, climate-resilient, and intelligent transport infrastructure, particularly concerning navigability along the Danube River and rail connectivity. Current infrastructure planning lacks integrated data analytics capabilities, resulting in suboptimal resource allocation, environmental impact assessments, and strategic decision-making.

\textbf{Assessment of Needs:} The target group—comprising Romanian and Bulgarian governmental bodies, transport authorities, and infrastructure agencies—requires advanced technological solutions to:
\begin{itemize}
    \item Analyze environmental impact of infrastructure projects on ecosystems and climate
    \item Optimize route planning for both rail and waterway transport
    \item Predict infrastructure maintenance needs using predictive analytics
    \item Integrate multimodal transport data for enhanced TEN-T network connectivity
    \item Support evidence-based policy decisions with real-time data visualization
    \item Ensure compliance with EU climate resilience and sustainability standards
\end{itemize}

\textbf{Challenges:} Key challenges include fragmented data sources, lack of cross-border data integration, limited AI/ML expertise in public sector organizations, and insufficient tools for scenario modeling and impact assessment. Our AI-powered platform addresses these gaps by providing an integrated, intelligent solution tailored to the specific requirements of cross-border infrastructure development.
 
 
\paragraph{Objectives} Make sure they (a) Respond to the selected priorities and (b) Respond to the needs or challenges specified above

\textbf{Objective 1:} Develop and deploy an AI-powered data analytics platform specifically designed for cross-border infrastructure planning, focusing on navigability and rail connectivity between Romania and Bulgaria, contributing to Specific Objective 3.2 (sustainable, climate-resilient, intelligent mobility).

\textbf{Objective 2:} Enhance decision-making capacity of Romanian and Bulgarian transport authorities through advanced data visualization, predictive modeling, and scenario analysis tools that integrate environmental, economic, and social impact assessments.

\textbf{Objective 3:} Improve access to TEN-T networks by optimizing intermodal connectivity between rail and waterway infrastructure through AI-driven route optimization and capacity analysis.

\textbf{Objective 4:} Build institutional capacity in AI and data analytics within cross-border public sector organizations through knowledge transfer, training programs, and establishment of sustainable data-sharing protocols.

\textbf{Objective 5:} Demonstrate measurable improvements in infrastructure planning efficiency, environmental sustainability, and cross-border cooperation through pilot implementation and comprehensive monitoring of key performance indicators
 

\paragraph{Activities}
Describe the activities that the project will undertake to produce the envisaged results

\textbf{Activity 1: Needs Assessment and Data Collection (Months 1-6)}

The project will begin with comprehensive stakeholder consultations across Romania and Bulgaria to understand the specific needs and challenges faced by transport authorities and infrastructure agencies. During this phase, we will systematically map existing data sources and infrastructure planning processes to identify gaps and opportunities for improvement. Working closely with partner organizations, we will define detailed technical requirements and establish measurable performance indicators that align with the project objectives. This foundational work will culminate in the establishment of formal data-sharing agreements and protocols that ensure secure, efficient cross-border information exchange throughout the project lifecycle.

\textbf{Activity 2: Platform Development and Customization (Months 4-18)}

Building on the needs assessment findings, our technical team will develop core AI and machine learning algorithms specifically designed for infrastructure analysis in the cross-border context. This involves creating an integrated database architecture capable of harmonizing and processing diverse data sources from both countries. We will build intuitive, user-friendly dashboards and visualization tools that enable non-technical users to access complex analytics insights easily. The platform will include sophisticated predictive maintenance and optimization modules that help authorities anticipate infrastructure needs and allocate resources efficiently. Throughout this development phase, we will conduct iterative testing and refinement sessions with end-users to ensure the platform meets practical requirements and delivers genuine value.

\textbf{Activity 3: Pilot Implementation (Months 12-24)}

Once the core platform capabilities are ready, we will deploy the system in carefully selected pilot regions and transport corridors that represent key priorities for Romania-Bulgaria connectivity. The platform will be applied to specific navigability projects along the Danube and targeted rail infrastructure initiatives, providing real-world testing of its analytical capabilities. During the pilot phase, we will continuously monitor system performance and actively gather user feedback to identify areas for improvement. The platform will also be used to conduct environmental impact assessments for proposed infrastructure projects, demonstrating its practical value in supporting sustainable development decisions.

\textbf{Activity 4: Capacity Building and Training (Months 6-30)}

Recognizing that technology is only as effective as the people who use it, we will design and deliver comprehensive training programs tailored to public sector staff at various skill levels. These programs will combine theoretical knowledge with hands-on practice, ensuring participants can confidently use the platform in their daily work. We will organize cross-border knowledge exchange workshops that bring together Romanian and Bulgarian professionals to share experiences and build lasting collaborative relationships. Supporting materials including detailed user manuals and technical documentation will be developed in multiple languages to ensure accessibility. To ensure long-term success, we will establish a dedicated helpdesk and ongoing support mechanisms that provide timely assistance as users encounter challenges or discover new application possibilities.

\textbf{Activity 5: Evaluation and Sustainability Planning (Months 24-36)}

In the final phase, we will conduct a comprehensive evaluation of the project's impact, measuring achievements against the established objectives and performance indicators. This process includes systematically documenting best practices and lessons learned throughout implementation, creating valuable knowledge assets for future initiatives. Working with all partners, we will develop a detailed sustainability and scaling strategy that ensures the platform continues to deliver value beyond the project funding period. The project findings and success stories will be actively disseminated to the wider stakeholder community through conferences, publications, and direct engagement, promoting broader adoption of AI-driven approaches in infrastructure planning across the EU.

\newpage
\paragraph{Project results}
tangible deliverables of the project (such as
curricula, pedagogical and youth work materials, open educational resources, IT tools,  studies, peer-learning methods, etc.).

\textbf{R1. AI-Powered Infrastructure Analytics Platform:} Fully functional cloud-based platform with modules for data integration, predictive analytics, route optimization, environmental impact assessment, and real-time visualization dashboards. Platform includes API for integration with existing systems.

\textbf{R2. Cross-Border Infrastructure Database:} Comprehensive integrated database containing navigability data (Danube River), rail network information, environmental parameters, traffic flows, and socio-economic indicators for the Romania-Bulgaria cross-border region.

\textbf{R3. Feasibility Studies and Impact Assessments:} Three detailed studies: (1) Analysis of current infrastructure gaps and optimization opportunities; (2) Environmental and climate resilience impact assessment of proposed interventions; (3) Cost-benefit analysis of AI-driven planning versus traditional methods.

\textbf{R4. Training Materials and Capacity Building Package:} Complete training curriculum including user manuals, video tutorials, hands-on workshops, e-learning modules, and technical documentation in Romanian, Bulgarian, and English. Certification program for platform administrators.

\textbf{R5. Policy Recommendations and Sustainability Framework:} Strategic document outlining policy recommendations for AI adoption in infrastructure planning, data governance framework for cross-border cooperation, and sustainability plan for long-term platform operation and maintenance.
 
\newpage
\paragraph{Key outcomes}
Make sure that the outcomes derive from the proposed Activities /Project Results and that, at the same time, answer the set objectives.

\textbf{Outcome 1: Enhanced Infrastructure Planning Efficiency}

The implementation of our AI-powered platform will significantly streamline infrastructure planning processes, reducing the time required for comprehensive analysis by 30-40\% through automated data processing and intelligent insights generation. Transport authorities will benefit from improved accuracy in their traffic flow predictions and capacity assessments, with expected improvements of around 25\% compared to traditional methods. Perhaps most importantly for cross-border cooperation, the platform will enhance coordination between Romanian and Bulgarian authorities by enabling data exchange that is 50\% faster than current manual processes, fostering more responsive and collaborative decision-making.

\textbf{Outcome 2: Improved Environmental Sustainability}

Environmental protection will be strengthened through the project's AI-powered impact modeling capabilities, which will reduce the time needed for environmental assessments by approximately 20\% while maintaining or improving accuracy. The platform's analytical power will enable the identification of climate-resilient infrastructure solutions that minimize carbon footprints and support long-term sustainability goals. By making environmental analysis more efficient and comprehensive, the project will facilitate better integration of environmental protection measures into infrastructure projects from the earliest planning stages, ensuring that sustainability considerations are embedded in all development decisions.

\textbf{Outcome 3: Strengthened Institutional Capacity}

The project will build lasting human capital by training at least 120 public sector professionals in AI and data analytics, creating a skilled workforce capable of leveraging advanced technologies for infrastructure planning. Beyond individual skill development, the project will establish four cross-border working groups that continue collaboration beyond the project's lifetime, creating permanent channels for knowledge exchange and joint problem-solving. The adoption of sustainable data-sharing protocols by partner institutions will ensure that the improved collaborative practices become embedded in organizational routines, creating a foundation for future cross-border initiatives.

\textbf{Outcome 4: Improved TEN-T Network Connectivity}

The platform's optimization capabilities will enable the identification and development of improved multimodal transport routes that effectively connect rail and waterway infrastructure, addressing current fragmentation in the transport network. Regional and local communities will benefit from enhanced accessibility to TEN-T core network corridors, reducing travel times and improving economic opportunities. By providing tools to analyze and optimize the integration of regional and local mobility systems with TEN-T networks, the project will help create a more coherent and efficient transport ecosystem that serves both international freight and passenger needs.

\textbf{Outcome 5: Evidence-Based Policy Development}

The availability of robust, real-time analytics will transform policy development, with data-driven strategic documents for infrastructure development being adopted by authorities as the new standard for planning. Decision-makers will achieve improved resource allocation through predictive analytics and scenario modeling that reveals the likely impacts of different investment choices before commitments are made. The project will also enhance transparency and stakeholder engagement in planning processes, as the platform's visualization capabilities make complex data accessible to diverse audiences, fostering more informed public discourse and building trust in infrastructure development decisions.

\newpage
\paragraph{Partners' profile} Describe any specific requirement e.g. type of organisation needed (NGO, VET, etc.), and/or country of establishment, and/or complementary skills that we are looking for in the partnership (e.g. expertise in IT tools or in working with children etc.)

\textbf{Lead Partner Requirements:}

Our startup will serve as the lead partner, bringing specialized expertise in AI, machine learning, and data analytics to the project consortium. As an IT Solutions company established in either Romania or Bulgaria, we possess the technical capacity necessary for cloud-based platform development and ongoing maintenance. Our prior experience in developing solutions for public sector and infrastructure projects positions us well to understand the unique requirements and constraints of governmental organizations, ensuring that the platform we develop will be both technically sophisticated and practically applicable to real-world planning challenges.

\textbf{Required Project Partners:}

\textbf{Partner 1: Romanian Transport Authority or Infrastructure Agency}

We seek a Romanian public sector organization with direct responsibility for transport infrastructure planning as our primary implementation partner. This organization must have access to relevant data sources and active involvement in decision-making processes, which are essential for effective platform development and testing. The partner should possess the institutional capacity to pilot the platform in real operational contexts and ultimately adopt it as part of their standard planning toolkit. Crucially, this partner must demonstrate a commitment to long-term sustainability, ensuring that the platform continues to deliver value well beyond the project funding period.

\textbf{Partner 2: Bulgarian Transport Authority or Infrastructure Agency}

The Bulgarian counterpart to our Romanian partner is essential for genuine cross-border coordination and cooperation. This organization should hold a similar mandate and possess comparable capacities to ensure balanced participation and mutual learning throughout the project. While not strictly required, prior experience in EU-funded projects would be highly valuable, as it demonstrates understanding of project management requirements and cross-border collaboration dynamics. This partner's engagement is critical for ensuring that the platform serves the needs of both countries equally and facilitates truly integrated infrastructure planning.

\textbf{Partner 3: Research Institution or University}

A research institution or university based in Romania or Bulgaria will bring crucial academic expertise to the project, particularly in fields such as transport engineering, environmental science, or data science. This partner will provide the analytical rigor needed for comprehensive impact assessment and evaluation research, ensuring that project claims are scientifically validated. Their location in the project region ensures local knowledge and context, while their experience in knowledge transfer and capacity building will be invaluable for developing effective training programs and ensuring that learning outcomes extend beyond immediate project participants.

\textbf{Partner 4: Environmental NGO or Sustainability Consultancy (Optional)}

While not mandatory, an environmental NGO or sustainability consultancy would significantly strengthen the project's environmental credentials and stakeholder engagement capacity. Such a partner would bring specialized expertise in environmental impact assessment and climate resilience, ensuring that these critical considerations are thoroughly integrated into platform design and application. Their deep knowledge of EU environmental standards and regulations would help ensure compliance and best practices, while their experience in stakeholder engagement and policy advocacy would enhance the project's ability to communicate with diverse audiences and influence policy development.

\textbf{Complementary Skills Needed:}

To maximize project effectiveness, the consortium should collectively possess expertise in Geographic Information Systems (GIS), which is fundamental for spatial analysis and visualization of infrastructure networks. Specialized knowledge of Danube navigation and inland waterway management is essential given the project's focus on navigability, while rail infrastructure technical expertise ensures that rail planning components are properly addressed. Strong project management capabilities and familiarity with EU funding mechanisms will be critical for smooth project execution and compliance with all requirements. Multilingual capabilities encompassing Romanian, Bulgarian, and English are necessary for effective communication and documentation across the partnership. Finally, excellent stakeholder engagement and communication skills will enable the consortium to work effectively with diverse governmental, technical, and public audiences throughout the project.
 
 




\newpage
\printbibliography
\nocite{*}

\end{document}
