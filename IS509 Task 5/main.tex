\documentclass{article}

% Language setting
% Replace `english' with e.g. `spanish' to change the document language
\usepackage[english]{babel}

% Set page size and margins
% Replace `letterpaper' with `a4paper' for UK/EU standard size
\usepackage[letterpaper,top=2cm,bottom=2cm,left=3cm,right=3cm,marginparwidth=1.75cm]{geometry}

% Useful packages
\usepackage{amsmath}
\usepackage{graphicx}
\usepackage[table]{xcolor}
\usepackage[colorlinks=true, allcolors=blue]{hyperref}

% Add these packages for the coding layout
\usepackage{multicol}
\usepackage{tikz}
\usetikzlibrary{calc}

\usepackage[
backend=biber,
style=ieee,
]{biblatex}


\addbibresource{sample.bib} %Imports bibliography file

% Import coding template
% Qualitative Coding Template
% Configurable column widths
\newlength{\LinePosition}
\newlength{\mainTextWidth}
\newlength{\codesWidth}
\newlength{\columnGap}

\setlength{\mainTextWidth}{0.70\textwidth}
\setlength{\LinePosition}{0.75\textwidth}
\setlength{\codesWidth}{0.25\textwidth}
\setlength{\columnGap}{0.05\textwidth}

% Usage: \codingSection{label}{interviewer_text}{participant_text}{codes}{notes}
\newcommand{\codingSection}[5]{
\paragraph{#1}
\noindent\begin{minipage}[t]{\mainTextWidth}
\subparagraph{Interviewer:} #2
\\
\subparagraph{Participant:} #3
\end{minipage}%
\hspace{\columnGap}%
\begin{minipage}[t]{\codesWidth}
\small
#4
\vspace{0.3cm}
\textit{Notes: #5}
\end{minipage}

\vspace{0.5cm}
}

% Header for coding section
\newcommand{\codingHeader}{
\vspace{0.4cm}
\noindent\begin{minipage}[t]{\LinePosition}
\textbf{Interview Transcript}
\end{minipage}%
\hspace{0.02\textwidth}%
\begin{tikzpicture}[remember picture, overlay]
\draw[thick, gray] (0,0) -- (0,-\textheight+2cm);
\end{tikzpicture}%
\hspace{0.01\textwidth}%
\begin{minipage}[t]{\codesWidth}
\textbf{Codes \& Remarks}
\end{minipage}

\vspace{0.5cm}
}


\begin{document}

% ---------------------------
% Title page (custom)
% ---------------------------
\begin{titlepage}
  \begin{center}
    \includegraphics[width=5cm]{nup_logo.png} \\[1cm]
    {\Large Neapolis University Pafos} \\[0.5cm]

    \begin{tabular}{@{}l@{}}
      {\large \textbf{Course Code:} IS509} \\[0.2cm]
      {\large \textbf{Course Title:} Research Methodology} \\[0.2cm]
      {\large \textbf{Audience Instructor:} Marios Touloupos}
    \end{tabular} \\[2cm]

    {\huge \textbf{Task 5: Analysis and Interpretation of Qualitative Data}} \\[1.5cm]

     \begin{tabular}{@{}l@{}}
      {\large \textbf{Name:} Aleksandr Petrunin} \\[0.3cm]
      {\large \textbf{Student ID:} 1251114137}
    \end{tabular} \\[2cm]

    {\large \today}
  \end{center}
\end{titlepage}



\section{Introduction: interview}

\paragraph{Interview topic:} Remote Work Experience During the Pandemic.

\paragraph{A)}
\subparagraph{Interviewer:} Can you describe how your work life changed when you started working from home
during the pandemic?
\subparagraph{Participant:} At first, it was honestly very strange. I missed the routine of going to the office,
getting coffee with colleagues, just... having that buzz around me. But over time, I started to
appreciate the flexibility. I could start earlier, take short breaks when needed, and it made
balancing family life much easier.

\paragraph{B)}
\subparagraph{Interviewer:} What aspects did you find most challenging?
\subparagraph{Participant:} The main challenge was staying focused. There were so many distractions — kids,
household chores, even just the temptation to relax. Also, communication with my team became
harder. We used video calls, but it wasn’t the same. Sometimes misunderstandings happened
because we couldn’t read each other’s body language.


\paragraph{C)}
\subparagraph{Interviewer:} How did your organization support you during that period?
\subparagraph{Participant:} They tried. We had weekly check-ins, and our manager encouraged us to take
mental health days. But I think they underestimated how exhausting constant online meetings
could be. I often felt drained by the end of the day.


\paragraph{D)}
\subparagraph{Interviewer:} Looking back, do you think remote work had more positive or negative effects
overall?
\subparagraph{Participant:} Hmm, I’d say mixed. Personally, it gave me more control over my schedule and
reduced stress from commuting. But socially, it made me feel isolated. I missed the informal
chats — that’s where a lot of creativity and problem-solving used to happen.


\paragraph{E)}
\subparagraph{Interviewer:} Would you prefer to continue working remotely in the future?
\subparagraph{Participant:} Probably a hybrid model. I think a few days at home and a few at the office would be
ideal. It keeps the flexibility but still allows that human connection.


\newpage
\section{Practicing qualitative coding}

\subsection{Open coding}
\textit{- fragments the data and assigns labels to capture relevant concepts}

\paragraph{Interview topic:} Remote Work Experience During the Pandemic.

\codingHeader

\codingSection{A)}{Can you describe how your work life changed when you started working from home during the pandemic?}{At first, it was honestly very strange. I missed the routine of going to the office, getting coffee with colleagues, just... having that buzz around me. But over time, I started to appreciate the flexibility. I could start earlier, take short breaks when needed, and it made balancing family life much easier.}{\textcolor{blue}{[Strange experience.]}\\ \textcolor{blue}{[Missing social interaction.]}\\ \textcolor{blue}{[Appreciation for flexibility.]}\\}{ ...}

\codingSection{B)}{What aspects did you find most challenging?}{The main challenge was staying focused. There were so many distractions — kids, household chores, even just the temptation to relax. Also, communication with my team became harder. We used video calls, but it wasn't the same. Sometimes misunderstandings happened because we couldn't read each other's body language.}{\textcolor{blue}{[Distractions at home.]}\\ \textcolor{blue}{[Communication difficulties.]}\\ \textcolor{blue}{[Lack of non-verbal cues.]}\\}{...}

\codingSection{C)}{How did your organization support you during that period?}{They tried. We had weekly check-ins, and our manager encouraged us to take mental health days. But I think they underestimated how exhausting constant online meetings could be. I often felt drained by the end of the day.}{\textcolor{blue}{[Organizational support efforts.]}\\ \textcolor{blue}{[Exhaustion from online meetings.]}\\}{...}

\codingSection{D)}{Looking back, do you think remote work had more positive or negative effects overall?}{Hmm, I'd say mixed. Personally, it gave me more control over my schedule and reduced stress from commuting. But socially, it made me feel isolated. I missed the informal chats — that's where a lot of creativity and problem-solving used to happen.}{\textcolor{blue}{[Mixed feelings about remote work.]}\\ \textcolor{blue}{[Increased schedule control.]}\\ \textcolor{blue}{[Social isolation.]}\\ \textcolor{blue}{[Informal chats fostered creativity and solved problems.]}\\}{...}

\codingSection{E)}{Would you prefer to continue working remotely in the future?}{Probably a hybrid model. I think a few days at home and a few at the office would be ideal. It keeps the flexibility but still allows that human connection.}{\textcolor{blue}{[Preference for hybrid model.]}\\ \textcolor{blue}{[Value of flexibility.]}\\ \textcolor{blue}{[Need for human connection.]}\\}{...}

\newpage
\subsection{Axial coding}
\textit{- relates codes (categories and subcategories) to each other, via a combination of inductive and deductive thinking}

\begin{multicols}{2}
\paragraph{Initial categories}
\begin{enumerate}
  \item \textbf{Work Environment Changes}
    \begin{enumerate}
      \item Strange experience
      \item Missing social interaction
      \item Appreciation for flexibility
    \end{enumerate}
  \item \textbf{Challenges of Remote Work}
    \begin{enumerate}
      \item Distractions at home
      \item Communication difficulties
      \item Lack of non-verbal cues
    \end{enumerate}
  \item \textbf{Organizational Support}
    \begin{enumerate}
      \item Organizational support efforts
      \item Exhaustion from online meetings
    \end{enumerate}
  \item \textbf{Impact on Well-being}
    \begin{enumerate}
      \item Mixed feelings about remote work
      \item Increased schedule control
      \item Social isolation
      \item Informal chats fostered creativity and solved problems
    \end{enumerate}
  \item \textbf{Future Work Preferences}
    \begin{enumerate}
      \item Preference for hybrid model
      \item Value of flexibility
      \item Need for human connection
    \end{enumerate}
\end{enumerate}

\columnbreak

\paragraph{Derived themes}
\begin{enumerate}
  \item \textbf{Adaptation and Transition}
    \begin{itemize}
      \item Adjusting to new work environment
      \item Evolving perception of remote work
    \end{itemize}
  \item \textbf{Social Connectivity}
    \begin{itemize}
      \item Loss of informal interactions
      \item Impact on collaboration and creativity
      \item Communication barriers
    \end{itemize}
  \item \textbf{Work-Life Balance}
    \begin{itemize}
      \item Flexibility benefits
      \item Home distractions
      \item Schedule autonomy
    \end{itemize}
  \item \textbf{Organizational Response}
    \begin{itemize}
      \item Support mechanisms
      \item Meeting fatigue
    \end{itemize}
  \item \textbf{Hybrid Work Vision}
    \begin{itemize}
      \item Balancing flexibility and connection
      \item Future work model preferences
    \end{itemize}
\end{enumerate}
\end{multicols}

\newpage
\paragraph{Relationships between categories}
\begin{itemize}
  \item The theme of \textbf{Adaptation and Transition} is influenced by the \textbf{Work Environment Changes} category, as individuals adjust to new routines and environments.
  \item \textbf{Social Connectivity} is directly related to the \textbf{Challenges of Remote Work} category, highlighting how communication difficulties and lack of non-verbal cues impact collaboration.
  \item The \textbf{Work-Life Balance} theme emerges from both the \textbf{Work Environment Changes} and \textbf{Challenges of Remote Work} categories, reflecting the dual nature of flexibility and distractions.
  \item \textbf{Organizational Response} is connected to the \textbf{Organizational Support} category, emphasizing the role of support mechanisms and the challenges of meeting fatigue.
  \item The \textbf{Hybrid Work Vision} theme is derived from the \textbf{Future Work Preferences} category, indicating a desire for a balanced approach to work.
\end{itemize}

\paragraph{Properties:} 
\begin{itemize}
  \item \textbf{Adaptation and Transition}
    \begin{itemize}
      \item \textit{Duration:} Gradual process over time (initial strangeness to eventual appreciation)
      \item \textit{Direction:} Negative to positive trajectory
      \item \textit{Intensity:} Strong initial resistance, moderate acceptance
    \end{itemize}
  \item \textbf{Social Connectivity}
    \begin{itemize}
      \item \textit{Frequency:} Reduced from daily to occasional interactions
      \item \textit{Quality:} Formal (structured meetings) vs. informal (spontaneous conversations)
      \item \textit{Impact:} High significance on creativity and problem-solving
    \end{itemize}
  \item \textbf{Work-Life Balance}
    \begin{itemize}
      \item \textit{Control:} Increased autonomy over schedule
      \item \textit{Boundaries:} Blurred between work and personal life
      \item \textit{Trade-offs:} Flexibility vs. distractions
    \end{itemize}
  \item \textbf{Organizational Response}
    \begin{itemize}
      \item \textit{Adequacy:} Partial support (good intentions, incomplete implementation)
      \item \textit{Awareness:} Limited understanding of remote work challenges
      \item \textit{Consequences:} Meeting fatigue and exhaustion
    \end{itemize}
  \item \textbf{Hybrid Work Vision}
    \begin{itemize}
      \item \textit{Balance:} Integration of remote and office work
      \item \textit{Priorities:} Flexibility and human connection equally valued
      \item \textit{Implementation:} Partial week split between locations
    \end{itemize}
\end{itemize}


\newpage
\subsection{Selective coding}
\textit{- identifies the core category, systematically relating it to other categories, validating those relationships, and filling in categories that need further refinement and development}

\paragraph{Core category:} \textbf{Navigating Remote Work: Balancing Flexibility and Connection}
\paragraph{Narrative:} The experience of remote work during the pandemic has been a complex journey of adaptation and transition. Initially, individuals faced a strange and unfamiliar work environment, missing the social interactions that once fueled creativity and collaboration. Over time, many began to appreciate the newfound flexibility that remote work offered, allowing for greater control over their schedules and improved work-life balance. However, this flexibility came with its own set of challenges. Distractions at home and communication difficulties, particularly the lack of non-verbal cues, often hindered productivity and team cohesion. Organizations attempted to support their employees through various initiatives, such as weekly check-ins and mental health days. Yet, these efforts sometimes fell short, leading to exhaustion from constant online meetings. The overall impact of remote work was mixed, with individuals recognizing both the benefits and drawbacks of this new mode of working. Looking ahead, there is a strong preference for a hybrid work model that combines the best of both worlds—offering the flexibility of remote work while maintaining the essential human connections fostered in the office.



\newpage
\subsection{Report}
Write a short report (600–800 words) explaining how you coded the data, how you
derived themes, and what insights they suggest.






\newpage
\printbibliography
\nocite{*}

\end{document}
