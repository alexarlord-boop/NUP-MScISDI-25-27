\documentclass{article}

% Language setting
% Replace `english' with e.g. `spanish' to change the document language
\usepackage[english]{babel}

% Set page size and margins
% Replace `letterpaper' with `a4paper' for UK/EU standard size
\usepackage[letterpaper,top=2cm,bottom=2cm,left=3cm,right=3cm,marginparwidth=1.75cm]{geometry}

% Useful packages
\usepackage{amsmath}
\usepackage{graphicx}
\usepackage[table]{xcolor}
\usepackage[colorlinks=true, allcolors=blue]{hyperref}

\usepackage[
backend=biber,
style=ieee,
]{biblatex}


\addbibresource{sample.bib} %Imports bibliography file

\begin{document}

% ---------------------------
% Title page (custom)
% ---------------------------
\begin{titlepage}
  \begin{center}
    \includegraphics[width=5cm]{nup_logo.png} \\[1cm]
    {\Large Neapolis University Pafos} \\[0.5cm]

    {\large \textbf{Course Code:} IS509} \\[2cm]

    {\huge \textbf{Task 4: Research Questions and Hypotheses}} \\[1.5cm]

    {\large \textbf{Name:} Aleksandr Petrunin} \\[0.3cm]
    {\large \textbf{Student ID:} 1251114137} \\[2cm]

    {\large \today}
  \end{center}
\end{titlepage}

\section{Introduction (Task 3 recap)}
In Task 3, we explored the Agent-to-Agent (A2A) communication framework, which facilitates interaction among autonomous agents in distributed systems. 
A2A's extension mechanism allows agents to dynamically adapt their communication protocols based on context and requirements. 
However, while A2A provides a flexible structure for agent interactions, it lacks a formal semantic layer that ensures deterministic communication and verifiable coordination among agents. 
This gap presents challenges in scenarios where agents must reliably delegate tasks, negotiate roles, or coordinate actions without ambiguity.

This gap raises critical questions: How can agents achieve deterministic communication within A2A's extension mechanism? What formal semantics are necessary for verifiable multi-agent coordination? Can a typed semantic layer maintain A2A's flexibility while enabling reliable task delegation and negotiation?


\section{Research Questions and Hypotheses}

\paragraph{Refined questions}
\textbf{Goal:} Clarity + Feasibility + Relevance.\\
\textbf{Strategy:} 
\subparagraph{Attempt 1}
\begin{enumerate}
  \item How can agents achieve deterministic communication within A2A's extension mechanism?
  \item What formal semantics are required  for verifiable multi-agent coordination?
  \item Can a typed semantic layer be integrated into A2A without compromising its adaptability?
\end{enumerate}

\subparagraph{Attempt 2}
\begin{enumerate}
  \item How can LLM-based agents achieve \textcolor{red}{deterministic} communication within A2A's extension mechanism?
  \item Will the integration of a typed semantic layer improve task delegation and negotiation among LLM-based agents?
  \item What \textcolor{red}{formal semantics} are required  for \textcolor{red}{verifiable and deterministic coordination} of the LLM-based agents?
  \item Can a typed semantic layer be integrated into A2A framework without \textcolor{red}{compromising its adaptability}?
  \item Will the proposed semantic layer affect small and large scale MAS \textcolor{red}{differently}?
  \item Will the proposed semantic layer affect MAS made of SLMs, LLMs, hybrid set \textcolor{red}{differently}?
\end{enumerate}

\subparagraph{Attempt 3-5}
\begin{enumerate}
  \item [Does it work?](Casual) To what extent does integration of a typed semantic layer in A2A bring determinism (measured by task completion consistency) in collaborative activities among LLM-based agents compared to standard A2A?
  \item [Why does it work?](Descriptive) What formal semantic constructs (e.g., type systems, truth conditions, predicates) are required to achieve deterministic task delegation and role negotiation in LLM-based multi-agent systems?
  \item [When does it work?](Comparative/relational) How does coordination overhead and task completion accuracy of the typed semantic layer scale with MAS size (2, 5, and 10 agents) in the A2A framework?
  \item [For whom does it work?](Comparative/relational) Does the effectiveness of the typed semantic layer differ between MAS using small language models ($<$7B parameters) versus foundation models ($>$70B parameters) in terms of protocol adherence and negotiation success rates?
\end{enumerate}


\subsection{Topic1: Does it work?}
\paragraph{Research Question:} To what extent does integration of a typed semantic layer in A2A bring determinism (measured by task completion consistency) in collaborative activities among LLM-based agents compared to standard A2A?
\paragraph{Hypothesis:} The integration of a typed semantic layer in A2A will enhance determinism in collaborative activities among LLM-based agents, leading to higher task completion consistency (at least 15\% increase) compared to standard A2A without a semantic layer.
\paragraph{IV:} Integration of a typed semantic layer in A2A (with vs. without).
\paragraph{DV:} Task completion consistency (measured by the percentage of successful task completions across multiple trials).
\paragraph{Explanation:} If we get a 15\% or more increase in task completions on some random MAS setup, it indicates that the typed semantics might actually help agents interpret and execute tasks more reliably. 
This would provide evidence to support the alternative hypothesis that a formal semantic layer enhances determinism in multi-agent collaborations.
This evidence would allow to further explore the specific semantic constructs that contribute to this improvement, guiding future enhancements to the A2A framework.
\textbf{Alternatively}, if no improvement is observed, it would suggest that other factors may be more critical in achieving deterministic communication among LLM-based agents. In this case we rethink out approach.

\newpage
\subsection{Topic2: Why does it work?}
\paragraph{Research Question:} What formal semantic constructs (e.g., type systems, truth conditions, predicates) are required to achieve deterministic task delegation and role negotiation in LLM-based multi-agent systems?
\paragraph{Hypothesis:} Deterministic task delegation and role negotiation in LLM-based multi-agent systems emerge when agents share a typed interaction protocol—comprising (1) a shared ontology of task types, (2) well-formed truth-conditional commitments, and (3) role-dependent inference rules—allowing messages to be parsed into unambiguous semantic actions.
\paragraph{IV:} The degree of semantic formalization in the agent communication protocol.
\begin{enumerate}
  \item Presence/absence of type annotations on messages
  \item Presence/absence of role predicates
  \item Presence/absence of truth-conditional constraints
  \item Level of ontology structure (none → flat → hierarchical)
\end{enumerate}
\paragraph{DV:} Determinism of delegation and negotiation outcomes. Measured by:
\begin{enumerate}
  \item Variance in task-assignment decisions across runs
  \item Success rate of resolving role conflicts
  \item Number of negotiation cycles required
  \item Agreement consistency across agents
\end{enumerate}
\paragraph{Explanation:} The study investigates whether LLM agents achieve more predictable coordination when their communication is constrained by explicit semantic formalisms. Without such constraints, agents rely on natural-language inference, which introduces ambiguity and stochastic interpretation. By adding elements such as type systems, role predicates, and truth-conditional commitments, agent messages become machine-interpretable semantic acts rather than free-form text. If determinism increases with stronger semantic structure, it would support the hypothesis that LLM-multi-agent systems require lightweight formal semantics to achieve reliable delegation and negotiation.

\newpage
\subsection{Topic3: When does it work?}
\paragraph{Research Question:}
\paragraph{Hypothesis:}
\paragraph{IV:}
\paragraph{DV:}
\paragraph{Explanation}
A short explanation (300–400 words) justifying how the questions/hypotheses align with
your research purpose and methodology.

\newpage
\subsection{Topic4: For whom does it work?}
\paragraph{Research Question:}
\paragraph{Hypothesis:}
\paragraph{IV:}
\paragraph{DV:}
\paragraph{Explanation}
A short explanation (300–400 words) justifying how the questions/hypotheses align with
your research purpose and methodology.







\newpage
\printbibliography
\nocite{*}

\end{document}
